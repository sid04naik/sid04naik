%% Use the "normalphoto" option if you want a normal photo instead of cropped to a circle
% \documentclass[10pt,a4paper,normalphoto]{altacv}
%% AltaCV uses the fontawesome5 and simpleicons packages.
%% See http://texdoc.net/pkg/fontawesome5 and http://texdoc.net/pkg/simpleicons for full list of symbols.


\documentclass[10pt,a4paper,ragged2e,withhyper]{altacv}
% Change the page layout if you need to
\geometry{left=1.25cm,right=1.25cm,top=1.5cm,bottom=1.5cm,columnsep=1.2cm}

% The paracol package lets you typeset columns of text in parallel
\usepackage{paracol}

% Change the font if you want to, depending on whether
% you're using pdflatex or xelatex/lualatex
% WHEN COMPILING WITH XELATEX PLEASE USE
% xelatex -shell-escape -output-driver="xdvipdfmx -z 0" mmayer.tex
\iftutex
  % If using xelatex or lualatex:
  \setmainfont{Lato}
\else
  % If using pdflatex:
  \usepackage[default]{lato}
\fi

% Change the colours if you want to
\definecolor{VividPurple}{HTML}{3E0097}
\definecolor{SlateGrey}{HTML}{2E2E2E}
\definecolor{LightGrey}{HTML}{666666}
% \colorlet{name}{black}
% \colorlet{tagline}{PastelRed}
\colorlet{heading}{VividPurple}
\colorlet{headingrule}{VividPurple}
% \colorlet{subheading}{PastelRed}
\colorlet{accent}{VividPurple}
\colorlet{emphasis}{SlateGrey}
\colorlet{body}{LightGrey}

% Change some fonts, if necessary
% \renewcommand{\namefont}{\Huge\rmfamily\bfseries}
% \renewcommand{\personalinfofont}{\footnotesize}
% \renewcommand{\cvsectionfont}{\LARGE\rmfamily\bfseries}
% \renewcommand{\cvsubsectionfont}{\large\bfseries}

% Change the bullets for itemize and rating marker
% for \cvskill if you want to
\renewcommand{\cvItemMarker}{{\small\textbullet}}
\renewcommand{\cvRatingMarker}{\faCircle}
% ...and the markers for the date/location for \cvevent
% \renewcommand{\cvDateMarker}{\faCalendar*[regular]}
% \renewcommand{\cvLocationMarker}{\faMapMarker*}

\begin{document}
\name{Siddhant Naik}
\tagline{Staff Software Engineer \& Backend Architect}
% Cropped to square from https://en.wikipedia.org/wiki/Marissa_Mayer#/media/File:Marissa_Mayer_May_2014_(cropped).jpg, CC-BY 2.0
%% You can add multiple photos on the left or right
\photoR{2.5cm}{OFFICE-PROFILE}
% \photoL{2cm}{Yacht_High,Suitcase_High}
\personalinfo{%
  % Not all of these are required!
  % You can add your own with \printinfo{symbol}{Testing}
  \email{contactsiddhantnaik@gmail.com}
  \phone{8830907549}
  \location{Bengaluru, India}
  % \homepage{marissamayr.tumblr.com}
  \linkedin{sid04naik}
  \github{sid04naik}
  % \stackoverflow{stackoverflow.com/users/4995949/siddhant}
  % \orcid{0000-0000-0000-0000} % Obviously making this up too.
  % \xtwitter{@marissamayer}
  %% You can add your own arbitrary detail with
  % \printinfo{}{stackoverflow}[stackoverflow.com/users/4995949/siddhant]
  % \printinfo{\faPaw}{Hey ho!}
  %% Or you can declare your own field with
  %% \NewInfoFiled{fieldname}{symbol}[optional hyperlink prefix] and use it:
  % \NewInfoField{gitlab}{\faGitlab}[https://gitlab.com/]
  % \gitlab{your_id}
	%%
  %% For services and platforms like Mastodon where there isn't a
  %% straightforward relation between the user ID/nickname and the hyperlink,
  %% you can use \printinfo directly e.g.
  % \printinfo{\faMastodon}{@username@instace}[https://instance.url/@username]
  %% But if you absolutely want to create new dedicated info fields for
  %% such platforms, then use \NewInfoField* with a star:
  % \NewInfoField*{mastodon}{\faMastodon}
  %% then you can use \mastodon, with TWO arguments where the 2nd argument is
  %% the full hyperlink.
  % \mastodon{@username@instance}{https://instance.url/@username}
}

\makecvheader

\cvsection{Profile}
\textbf{Backend Engineering Specialist}, Leadership-oriented technologist with 10+ years of experience in backend engineering, specializing in Go, Node.js, cloud platforms, Docker, NoSQL, caching, and queues. Passionate about building scalable systems, driving platform reliability, and mentoring high-performing teams. Available with a 15-day notice.

%% Depending on your tastes, you may want to make fonts of itemize environments slightly smaller
\AtBeginEnvironment{itemize}{\small}

%% Set the left/right column width ratio to 6:4.
\columnratio{0.6}

% Start a 2-column paracol. Both the left and right columns will automatically
% break across pages if things get too long.
\begin{paracol}{2}

\cvsection{Experience}

\cvevent{Staff Engineer}{Albert Invent}{May 2024 -- Present}{Bengaluru, IN}
\begin{itemize}
\item Designed multi-language support for customer data, enabling dynamic variations across regions and boosting global adoption by 40\%.
\item Implemented a scheduler with RRule and AWS EventBridge, reducing manual intervention by 70\%.
\item Built an async worker API, increasing throughput 5x, cutting processing from hours to minutes, and eliminating frontend freezes.
\item Delivered a Digital Signature and Approval system with versioning, ensuring 100\% audit compliance and reducing turnaround by 25\%.
\item Built a real-time notification system, improving user engagement.
\item \textbf{Tech Stack:} Nodejs, AWS, Microservices, REST API, DynamoDB, EventBridge, SQS, Lambda, Docker, K6
\end{itemize}

\divider

\cvevent{Senior Engineer}{Yellow.ai}{Oct 2022 -- Apr 2024}{Bengaluru, IN}
\begin{itemize}
\item Designed and launched an in-house URL Shortener Service, replacing third-party tools and enabling integrated analytics; drove 1M+ links/month and boosted upsell conversion by 40\%.
\item Built a high-throughput OTP SMS service, reducing bot response latency by 10\% and cutting external SMS costs.
\item Contributed to scalable, low-latency microservices with Go, Redis, RabbitMQ, and ElasticSearch.
\item \textbf{Tech Stack:} Nodejs, Go, Redis, ElasticSearch, RabbitMQ, Microservices, REST API
\end{itemize}

\divider

\cvevent{Software Engineer Specialist}{Kaleyra}{Dec 2020 -- Oct 2022}{Bengaluru, IN}
\begin{itemize}
\item Scaled SMS throughput from 1K to 5K TPS by migrating monolith to microservices and optimizing with Go.
\item Implemented DB sharding for horizontal scaling and latency reduction across distributed workloads.
\item Developed personalized SMS flows using AWS Serverless stack (Lambda, SQS, SNS, etc.), increasing customer engagement via custom payloads.
\item \textbf{Tech Stack:} Go, Nodejs, Redis, MongoDB, RabbitMQ, Microservices, REST API, New Relic, Graylog, Docker, Kubernetes, AWS
\end{itemize}

\newpage

\cvevent{Senior Software Engineer}{Sapnagroup}{June 2016 -- Dec 2020}{Panjim, IN}

\divider

\cvevent{Software Engineer}{OPSPL}{June 2015 -- June 2016}{Verna, IN}

\divider

\cvevent{Software Engineer - Intern }{Townhub Innovative solutions Pvt. Ltd}{Jan 2015 -- June 2015}{Mangalore, IN}


\cvsection{Publications}
\cvachievement{\faNewspaper}{IJRCCE}{Protein Structure and Function Prediction Using Machine Learning Methods}


% \cvsection{Most Proud of}

% \cvachievement{\faTrophy}{Courage I had}{to take a sinking ship and try to make it float}

% \divider

% \cvachievement{\faHeartbeat}{Persistence \& Loyalty}{I showed despite the hard moments and my willingness to stay with Yahoo after the acquisition}

% \divider

% \cvachievement{\faChartLine}{Google's Growth}{from a hundred thousand searches per day to over a billion}

% \divider

% \cvachievement{\faFemale}{Inspiring women in tech}{Youngest CEO on Fortune's list of 50 most powerful women}



% \cvsection{A Day of My Life}
% \hspace*{-5em}
% \wheelchart{1.5cm}{0.5cm}{%
% 10/11em/accent!30/Morning routine and exercise,
% 5/13em/accent!40/Breakfast with family,
% 15/10em/accent!50/Email review and planning,
% 20/8em/accent!70/Team meetings and collaboration,
% 20/8em/accent!90/Strategic planning and issue resolution,
% 10/8em/accent!60/Evening tea and reflection,
% 10/10em/accent!40/Pet walk and wind-down,
% 10/13em/accent!30/Family time and rest
% }

% use ONLY \newpage if you want to force a page break for
% ONLY the currentc column
% \newpage

%% Switch to the right column. This will now automatically move to the second
%% page if the content is too long.
\switchcolumn

\cvsection{Skills}
\cvtag{Go}
\cvtag{JavaScript}
\cvtag{Nodejs}
\cvtag{Firestore}
\cvtag{MongoDB}
\cvtag{Postgres}
\cvtag{DynamoDB}
\cvtag{MySQL}
\cvtag{Redis}
\cvtag{AWS}
\cvtag{Git}
\cvtag{SVN}
\cvtag{RabbitMQ}
\cvtag{NewRelic}
\cvtag{Graylog}
\cvtag{Kubernetes}
\cvtag{Docker}
\cvtag{Unix}
\cvtag{K6}
\cvtag{CI/CD}
\cvtag{Serverless}

\cvsection{Certification}

\cvachievement
{\faCode} % Go icon approximation (custom symbol or text fallback since no official FontAwesome icon for Go)
{\textbf{\href{http://tinyurl.com/go-certificate}{Go Developer's Guide}}}{Udemy}
{Built scalable back-end services in Go using Goroutines, Channels, Interfaces, and REST APIs.}

\divider

\cvachievement
{\faCloud} % Use a custom icon or fallback to \faCloud if \faAws is not available
{\textbf{\href{http://tinyurl.com/aws-serverless-certificate}{AWS Serverless APIs and APPs}}}{Udemy}
{Developed serverless apps using AWS Lambda, API Gateway, and DynamoDB with automated deployment workflows.}

\divider

\cvachievement
{\faNodeJs} % If not available, fallback to \faCode or \faJs
{\textbf{\href{http://tinyurl.com/nodejs-mongo-certificate}{Node.js, Express MongoDB}}}{Udemy}
{Developed REST APIs using Node.js, Express, and MongoDB with auth and production setup.}

\divider

\cvachievement
{\faLock} % Security-oriented icon for incident response
{\textbf{\href{http://tinyurl.com/incident-responder-certificate}{Certified Incident Responder}}}{PagerDuty}
{Learned incident handling, threat detection, and breach mitigation techniques.}

\cvsection{Education}

\cvevent{MSc.\ Software Technology (ST)}{AIMIT}{2013 -- 2015}{Mangalore University, KA}

\divider

\cvevent{Bachelor of Computer Application}{AIMIT}{2013 -- 2015}{Goa University, GA}

\newpage

\cvsection{Languages}

\cvskill{English}{5}
\cvskill{Hindi}{4}
\cvskill{Konkani}{5}
\cvskill{Marathi}{3}


% \cvsection{Strengths}

% % Don't overuse these \cvtag boxes — they're just eye-candies and not essential. If something doesn't fit on a single line, it probably works better as part of an itemized list (probably inlined itemized list), or just as a comma-separated list of strengths.

% % The `ragged2e` document class option might cause automatic linebreaks between \cvtag to fail.
% % Either remove the ragged2e option; or
% % add \LaTeXraggedright in the paragraph for these \cvtag
% {\LaTeXraggedright
% \cvtag{Hard-working}
% \cvtag{Hard-working}
% \cvtag{Eye for detail}
% \cvtag{Motivator \& Leader}
% \par}

% \divider\smallskip

% \cvtag{UX}
% \cvtag{Mobile Devices \& Applications}
% \cvtag{Product Management \& Marketing}






\cvsection{Referees}

% \cvref{name}{email}{mailing address}
\cvref{Mr.\ Aditya Nalla}{Albert Invent}{aditya.nalla@albertinvent.com}{}

\divider
\cvref{Mr.\ Sagar Palyekar}{Kaleyra, Now Tata Communications}{sagar.s.palyekar@gmail.com}{}



\end{paracol}

\end{document}